\section{Reward Machines}

\begin{notes}
    (15-20 pages)

    papers:
    \begin{itemize}
        \item original reward machine paper from 2018 \cite{RM2018}
        \item connection to LTL from 2019 \cite{LTL2019}
        \item newer reward machine paper from 2022 \cite{RM2022}
    \end{itemize}

    include updated RL framework graph to better differentiate that reward machines are not in the environment black box

    include same task from previous chapter and how a reward machine for that might look like as a graph

    central theorems:
    \begin{itemize}
        \item construction/correspondence to LTL and such
        \item convergence proof
    \end{itemize}

\end{notes}


\usetikzlibrary{graphs, quotes}
\begin{figure}[h]
    \centering
    \tikz{
    \node[draw, rectangle, minimum width=5em, minimum height=3em] (a) at (0,0) {Agent};
    \node[draw, rectangle, minimum width=5em, minimum height=3em] (e) at (6,0) {Environment};
    \node[draw, rectangle, minimum width=5em, minimum height=3em] (rm) at (3,-3) {Reward Machine};

    \graph[multi, grow right sep]{
    (a) ->["action $a_t$", bend left] (e);
    (e) ->["state $s_{t+1}$", bend left, swap] (a);
    (e) ->["state $s_{t+1}$", bend left] (rm);
    (rm) ->["reward $r_{t+1}$", bend left=30, pos=0.3, swap, auto] (a);
    (rm) ->["RM state $u_{t+1}$", bend left=50, pos=0.5, auto] (a);
    };
    }
    \caption{Diagram of the reinforcement learning feedback loop with a reward machine. The agent now not only gets an environment state but also a reward machine state.}
    \label{fig:rm_framework}
\end{figure}

The mathematical model for this is ...

\begin{definition}
    \label{def:mdprm}
    A \emph{Markov Decision Process with Reward Machine (MDPRM)} is ...
\end{definition}
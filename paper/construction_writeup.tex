\documentclass[12pt, a4paper]{article}

\usepackage{amsmath}
\usepackage{amssymb}
\usepackage{fontawesome5}
\usepackage{dsfont}

\usepackage{tikz}

\usepackage[backend=biber, sorting=ynt]{biblatex}
\addbibresource{sources.bib}

\newcommand{\defn}[1]{\textbf{Definition.}\ \emph{#1}}
\newcommand{\lem}[1]{\textbf{Lemma.}\\\emph{#1}}
\newcommand{\thm}[1]{\textbf{Theorem.}\\\emph{#1}}
\newcommand{\cor}[1]{\textbf{Corollary.}\\\emph{#1}}
\newcommand{\prf}[1]{\textbf{Proof.}\\#1}

\renewcommand{\phi}{\varphi}

\title{Reward Machine Construction Writeup}
\author{Maximilian Stollmayer}
\date{}

\begin{document}

\maketitle

\section*{Basic Definitions}

\defn{
	\emph{Propositional symbols} are statements that are either true or false. \emph{Formulas} over propositional symbols consist of combinations of them with operations $\neg$, $\wedge$, $\vee$, $\implies$ and $\iff$. We say a formula $\psi$ \emph{provable} from a formula $\phi$, if $\psi$ can be derived from $\phi$ and write $\phi \vdash \psi$.
}

In the following we will suppose that we are in a reinforcement learning setting with a finite set of states $S$, with $s_0$ being the initial state, a set $T \subseteq S$ of terminal states and a finite set of actions $A$. Furthermore we suppose that we have a finite set of propositional symbols $\mathcal{P}$.

\defn{
	A \emph{labeling function} $L : S \times A \times S \to 2^{\mathcal{P}}$ maps experiences to truth assigments over propositional symbols $\mathcal{P}$.
}

\defn{
	A \emph{Non-Markovian Reward Decision Process (NMRDP)} is a tuple $(S, A, s_0, T, R, \gamma)$, where $S, A, s_0, T$ and $\gamma$ are defined as in a regular MDP and $R : (S \times A)^+ \times S \to \mathbb{R}$ is a non-Markovian reward function that maps finite state-action histories into a real value. Note that $X^+ := \bigcup_{n=1}^{\infty} X^n$ represents all non-empty finite sequences of a set $X$.
}

\section*{Reward Machines}

\defn{
	A \emph{Mealy machine} is a tuple $(U, u_0, \Sigma, \mathcal{R}, \delta, \rho)$, where
	\begin{itemize}
		\item $U$ is a finite set of states
		\item $u_0 \in U$ is the initial state
		\item $\Sigma$ is a finite input alphabet
		\item $\mathcal{R}$ is a finite output alphabet
		\item $\delta : U \times \Sigma \to U$ is the transition function
		\item $\rho : U \times \Sigma \to \mathcal{R}$ is the output function
	\end{itemize}
}

\defn{
	A \emph{reward machine (RM)} is a Mealy machine $(U, u_0, \Sigma = 2^\mathcal{P}, \mathcal{R}, \delta, \rho)$, where $\mathcal{R}$ is a finite set of reward functions $S \times A \times S \to \mathbb{R}$.
}

\defn{
	The non-Markovian reward function $R$ \emph{induced by an RM} $(U, u_0, 2^\mathcal{P}, \mathcal{R}, \delta, \rho)$ is
	\begin{align*}
		R : (S \times A)^+ \times S                       & \to \mathbb{R}                                                     \\
		(s_0, a_0), \dots, (s_n, a_n), s_{n+1}            & \mapsto \rho\big(u_n, L(s_n, a_n, s_{n+1})\big)(s_n, a_n, s_{n+1}) \\
		R\big((s_0, a_0), \dots, (s_n, a_n), s_{n+1}\big) & = r(s_n, a_n, s_{n+1})
	\end{align*}
	where $u_n = \delta\big(u_{n-1}, L(s_{n-1}, a_{n-1}, s_n)\big)$ is defined recursively with the base case being the initial state $u_0$.
}

\section*{Logics and Automata}

ltl \& co, dfa, dfa construction theorem and proof (source?)

\defn{
	A \emph{deterministic finite automaton (DFA)} is a tuple $(U, u_0, \Sigma, \delta, F)$, where
	\begin{itemize}
		\item $U$ is a finite set of states
		\item $u_0 \in U$ is the initial state
		\item $\Sigma$ is a finite input alphabet
		\item $\delta : U \times \Sigma \to U$ is the transition function
		\item $F \subseteq U$ is a set of accepting states
	\end{itemize}
}

\section*{Reward Specifications}

\defn{
	% WARN: R is also used for the RM induces non-Markovian reward function!!!
	% TODO: maybe use hat or something (hat also used below)
	% TODO: define regular language above?
	A \emph{reward specification} is a set $R = \{(r_1, \phi_1), \dots, (r_N, \phi_N)\}$, where each $r_i \in \mathbb{R}$ and $\phi_i$ is a formula over the propositional symbols $\mathcal{P}$ expressed in some regular language.
}

\defn{
	Let $\tau = \big( (s_0, a_0), \dots, (s_n, a_n), s_{n+1} \big)\in (S \times A)^+ \times S$ be a trace of experiences. We say that the projection of the experiences of $\tau$ by $L$ \emph{entails} a formula $\phi$, and write $\tau \vdash_L \phi$, if $L(s_0, a_0, s_1) \dots L(s_n, a_n, s_{n+1}) \vdash \phi$.
}

\defn{
	The non-Markovian reward function $\hat{R}$ \emph{induced by the reward specification} $R = \{(r_1, \phi_1), \dots, (r_n, \phi_n)\}$ assigns reward $\hat{R}(\tau) := \sum_{k=1}^N \mathds{1}(\tau \vdash_L \phi_k)$ to a trace $\tau = \big( (s_0, a_0), \dots, (s_n, a_n), s_{n+1} \big) \in (S \times A)^+ \times S$.
}

\section*{Construction Theorem}

\thm{
	There exists a reward machine that induces the same non-Markovian reward function as a given reward specification $R = \{(r_1, \phi_1), \dots, (r_N, \phi_N)\}$.
}

\prf{
	Let...
}

\cor{

}

\section*{Example}

go through the construction of a very simple formula to reward machine

\printbibliography

\end{document}
